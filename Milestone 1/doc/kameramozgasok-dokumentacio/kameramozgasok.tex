\documentclass[10pt,a4paper]{article}
\usepackage[utf8x]{inputenc}
\usepackage{ucs}
\usepackage{amsmath}
\usepackage{amsfonts}
\usepackage{amssymb}
\usepackage{listings}
\author{Bordé Sándor}
\date{2011. május 11.}
\title{Kameramozgások}
\begin{document}
\maketitle
\paragraph*{}
Az alábbiakban a Fejlett grafikai algoritmusok gyakorlatra készített projektmunk
ánkban, a billiárd játékban megvalósított kameramozgásokat és az azokhoz kapcsolódó
 algoritmusokat, elméleti hátteret fogom ismertetni.
\section*{Mozgatás a tengelyek mentén}
\paragraph*{}
Számítógépes grafikában a kamera mozgáskor valójában nem a kamera helyzete változik, hanem a színtér transzformálódik el az ellentétes irányba. Így a tengelyek mentén történő mozgást a megfelelő tengelyek mentén történő eltolással lehet megvalósítani.
\paragraph*{}
A billiárd játékunkban mindhárom tengely mentén lehet haladni az FPS\footnote{First Person Shooter} játékokból ismert billentyűkkel: A és D vízszintesen balra-jobbra, W és S előre hátra. Függőleges irányban a T és G gombokkal tudunk fel illetve le menni. A mozgásokra be van állítva egy-egy ellenőrző feltétel, ami megakadályozza, hogy kimenjünk a szobából. Ezeket a feltételeket a szoba dimenziói adják.
\subsection*{Automatikus mozgatás}
\paragraph*{}
A programban megvalósítottam a folyamatos mozgatást. Ez valójában egy paraméterezhető függvény, amely paraméterben a kívánt kamerahelyzet koordinátáit várja (\textit{x}, \textit{y} és \textit{z}). A függvény ezután odamozgatja a kamerát. Mindhárom koordináta egyszerre változik, így valójában a légvonalban legrövidebb úton közlekedik. Ha valamelyik koordináta már megegyezik a kívánt értékkel, (pl. függőlegesen már egy vonalban vele), akkor az onnantól kezdve változatlan marad.
\paragraph*{}
A szebb látvány érdekében ez a mozgás időzített: adott időintervallum alatt teszi meg a távot, köbös csillapítással számítva a pillanatnyi sebességet. A köbös csillapítás képlete:\footnote{Az alábbi képleten alapul: http://www.gizma.com/easing/}

\begin{center}
\textit{v = 2 $\cdot$ $ t^3 $ + 2}, ha \textit{t} \textless 1 

\textit{v = 2 $\cdot$ $(t-2)^2$ $\cdot$ t + 2} , ha \textit{1 \textless t \textless 2}
\end{center}
A \textit{v} a pillanatnyi sebesség, \textit{t} az aktuális időpillanat, a 2 pedig a kezdősebesség illetve a növekmény mértéke is egyben (változtatható).
\paragraph*{}
A demo programban a \textit{z} billentyűre van kötve ez a funkció, ami alapértelmezetten az origóba mozgatja a kamerát.
\section*{Forgatás hagyományos módon}
A programban az összehasonlítás miatt két féle forgatást is implementáltam. Az egyik a hagyományos forgatás, a másik pedig a kvaterniókkal való forgatás.
Hagyományos forgatás esetén egyszerűen csak megváltoztatom az adott tengely körül való elforgatás szögét, majd a színtért ennek megfelelően transzformálom.
\subsection*{Manuális forgatás}
\paragraph*{}
A forgatóműveletek elérhetők a billentyűzetről is: a $\leftarrow$ és a $\rightarrow$ gombokkal a függőleges tengely mentén lehet forgatni, a $\uparrow$ és $\downarrow$ billentyűkkel pedig előre-hátra lehet dönteni a kamerát. A harmadik tengely mentén szándékosan nem lehet forgatni.
\subsection*{Időzített forgatás}
\paragraph*{}
A forgatás is meg lett valósítva időzítve, hasonlóan a mozgatáshoz. A forgás sebességét itt is a fent bemutatott csillapító függvény számítja ki.
A megvalósító függvény három paramétert vár: a forgatás szögét, a forgatás sebességét (adott időegység alatt hány fokkal forduljon) valamint a forgatás tengelyét.
\paragraph*{}
A demo programban az \textit{r} billentyűre van kötve a funkció, ez 90 fokkal forgatja el a kamerát.
\section*{Forgatás kvaterniók használatával}
\paragraph*{}
A kamera forgatásának másik módja a kvaterniókkal történő forgatás. Látványban nem különbözik a hagyományos forgatástól, viszont könnyebben és olvashatóbban adható meg a művelet. Hogy a programban használni tudjuk a kvaterniókat, implementáltam egy 3 dimenizós vektor osztályt (\textit{Vektor3}) és egy Kvaternió
 (\textit{Kvaternio}) osztályt.
\paragraph*{}
(A kvaternió elméleti hátterének leírásától most eltekintek, figyelembe véve, hogy az előadásanyagban illetve az interneten fellelhető számos forrásban le van írva.)
\subsection*{Kvaternió osztály}
\paragraph*{}
A Kvaternió osztályom implementálja a kvaternióval kapcsolatos alapműveleteket (konjugálás, normalizálás, két kvaternió szorzata), vektor forgatását kvaternióval, valamint a kvaternió paramétereinek beállítását forgatási tengelyből és forgatási szögből. Lehetőség van a kvaternió mátrixának \footnote{http://en.wikipedia.org/wiki/Quaternion} lekérésére. Ez utóbbit majd a színtér transzformálásánál használjuk fel.
\subsection*{Színtér transzformációja kvaterniókkal}
\paragraph*{}
A színtér kialakításában egy logikai változóval (a demo programban \textit{k} billentyű lenyomásával váltható) szabályozható, hogy hagyományos vagy kvaterniókkal végzett forgatást szeretnénk használni. A hagyományos forgatásról volt szó az előző részben.
\paragraph*{}
Forgatáskor megállapítjuk, hogy az egyes tengelyek mentén hány fokos forgatásra van szükség, majd ezekből létrehozzuk a két kvaterniót (a tengelyből és a szögből, ahol a tengelyt egy három dimenziós vektor reprezentálja). A két kvaterniót összeszorozzuk, ez jelenti a forgatás egymás utáni végrehajtását. A szorzat kvaternióból mátrixot készítünk. Ez lesz a modellnézeti mátrix, ami a \textit{glMultMatrix()} függvénynek átadva a megfelelő irányba transzformálja a színteret.
\subsection*{Időzített forgatás kvaterniókkal}
A hagyományos forgatás esetén a forgatás pillanatnyi sebességét egy csillapítási függvénnyel számítottam ki. Kvaterniókkal ez a művelet egyszerűsödött. Az úgynevezett SLERP\footnote{Spherical linear interpolation} műveletet hívtam segítségül a végrehajtáshoz.\footnote{http://en.wikipedia.org/wiki/Slerp}
\subsection*{Slerp}
\paragraph*{}
A spherical linear interpolation (gömbmenti lineáris interpoláció) egy kvaternióktól független geometriai művelet. Az alapelve az az, hogy egy körív minden pontja előáll az ív két végpontjának lineáris kombinációjaként. A Slerp képlete:
\begin{center}
$Slerp(p_0, p_1, t) = \frac{\sin{[(1-t)*\Omega]}}{\sin{\Omega}}p_0 + \frac{\sin{t\Omega}}{sin{\Omega}}p_1 $
\end{center}
,ahol \textit{$\Omega$} a forgatás szöge, és $0 \le \textit{t} \le 1$
\paragraph*{}
Forgatáskor létrehozunk két kvaterniót: az egyiket az aktuális szögelfordulásból, másikat pedig az elfordulás mértékével növelt értékből. Tengelynek azt kell választani, amely mentén forgatni szeretnénk (a demo programban a \textit{r} gomb a függőleges tengely szerint forgat). Ezután az időzítő által meghívott függvényben alkalmazzuk a fent leírt képletet, ami meghatározza a $q_m$ köztes kvaterniót. A kvaternió \textit{w} koordinátájának koszinuszából kapjuk meg az új elfordulási szöget.
\paragraph*{}
A forgatás sebességét az időpillanatok számával tudjuk szabályozni: minél kisebb a \textit{t} változó, annál lassabb a forgatás.
\pagebreak
\subsection*{A Slerp forráskódja a programomból}
\begin{lstlisting}

	/*Kezdeti es vegkvaterniok letrehozasa*/
	Vektor3 z(0.0f, 1.0f, 0.0f);		
	start = new Kvaternio();
	start->kvaternioTengelybolEsSzogbol(z, (zRot));
	start->normalizal();
	end = new Kvaternio();
	end->kvaternioTengelybolEsSzogbol(z, (zRot+withDeg));	
	end->normalizal();	
		
	/*Interpolacio*/
	float gt = rotateTime / 50.0;
	
	float cosQTheta = start->kvaternioCosTheta(*end);	
	float qTheta = std::acos(cosQTheta);				
	if (qTheta<0.0) qTheta=-qTheta;

	float sinQTheta = std::sqrt( 1.0 - cosQTheta * cosQTheta );	
	float sinTQTheta = std::sin(gt * qTheta) / sinQTheta;
	float sinMQTheta = std::sin((1.0 - gt) * qTheta) / sinQTheta;
	
	Kvaternio qm;
	qm.setX( start->getX()*sinMQTheta + end->getX()*sinTQTheta );
	qm.setY( start->getY()*sinMQTheta + end->getY()*sinTQTheta );
	qm.setZ( start->getZ()*sinMQTheta + end->getZ()*sinTQTheta );
	qm.setW( start->getW()*sinMQTheta + end->getW()*sinTQTheta );

	zRot = radianFokba( 2.0 * std::acos(qm.getW()) );	
\end{lstlisting}

\section*{Források, felhasznált információk}

\begin{itemize}
\item Csillapító függvény - http://www.gizma.com/easing/
\item Kvaterniók általános leírása - http://en.wikipedia.org/wiki/Quaternion
\item Kvaterniók leírása - http://www.geometrictools.com/Documentation/Quaternions.pdf
\item Kvaterniók használata számítógépes grafikában - \linebreak http://gpwiki.org/index.php/OpenGL:Tutorials:Using\_Quaternions\_to\_represent\_rotation
\item Slerp - http://en.wikipedia.org/wiki/Slerp
\item Kvaterniók, kameramozgások - előadásjegyzet, kurzus jegyzet
\end{itemize}
\end{document} 
